%%%%%%%%%%%%%%%%%
% This is an sample CV template created using altacv.cls
% (v1.1.5, 1 December 2018) written by LianTze Lim (liantze@gmail.com). Now compiles with pdfLaTeX, XeLaTeX and LuaLaTeX.
%
%% It may be distributed and/or modified under the
%% conditions of the LaTeX Project Public License, either version 1.3
%% of this license or (at your option) any later version.
%% The latest version of this license is in
%%    http://www.latex-project.org/lppl.txt
%% and version 1.3 or later is part of all distributions of LaTeX
%% version 2003/12/01 or later.
%%%%%%%%%%%%%%%%

%% If you need to pass whatever options to xcolor
\PassOptionsToPackage{dvipsnames}{xcolor}

%% If you are using \orcid or academicons
%% icons, make sure you have the academicons
%% option here, and compile with XeLaTeX
%% or LuaLaTeX.
% \documentclass[10pt,a4paper,academicons]{altacv}

%% Use the "normalphoto" option if you want a normal photo instead of cropped to a circle
% \documentclass[10pt,a4paper,normalphoto]{altacv}

\documentclass[10pt,a4paper,ragged2e]{altacv}

%% AltaCV uses the fontawesome and academicon fonts
%% and packages.
%% See texdoc.net/pkg/fontawecome and http://texdoc.net/pkg/academicons for full list of symbols. You MUST compile with XeLaTeX or LuaLaTeX if you want to use academicons.

% Change the page layout if you need to
\geometry{left=1cm,right=9cm,marginparwidth=6.8cm,marginparsep=1.2cm,top=1.25cm,bottom=1.25cm}

% Change the font if you want to, depending on whether
% you're using pdflatex or xelatex/lualatex
\ifxetexorluatex
  % If using xelatex or lualatex:
  \setmainfont{Lato}
\else
  % If using pdflatex:
  \usepackage[utf8]{inputenc}
  \usepackage[T1]{fontenc}
  \usepackage[default]{lato}
\fi

% Change the colours if you want to
\definecolor{Mulberry}{HTML}{72243D}
\definecolor{SlateGrey}{HTML}{2E2E2E}
\definecolor{LightGrey}{HTML}{666666}
\colorlet{heading}{Sepia}
\colorlet{accent}{Mulberry}
\colorlet{emphasis}{SlateGrey}
\colorlet{body}{LightGrey}

% Change the bullets for itemize and rating marker
% for \cvskill if you want to
\renewcommand{\itemmarker}{{\small\textbullet}}
\renewcommand{\ratingmarker}{\faCircle}

%% sample.bib contains your publications
\addbibresource{sample.bib}

\begin{document}
\name{Victor Afonso dos Reis}
\tagline{}
\photo{2.5cm}{vdosreis}
\personalinfo{%
  % Not all of these are required!
  % You can add your own with \printinfo{symbol}{detail}
  \email{victor.afonsoreis35@gmail.com}
  \phone{+55 (17)982204427}
  \mailaddress{Av Belvedere, 750, QD A Lote 16, 15056030}
  \location{São José do Rio Preto-SP, Brazil}
  %\homepage{}
 %\twitter{}
  \linkedin{www.linkedin.com/in/vdosreis}
  %\github{}
  %% You MUST add the academicons option to \documentclass, then compile with LuaLaTeX or XeLaTeX, if you want to use \orcid or other academicons commands.
  % \orcid{orcid.org/0000-0000-0000-0000}
  \birthday{19/Apr/1996}
}

%% Make the header extend all the way to the right, if you want.
\begin{fullwidth}
\makecvheader
\end{fullwidth}

%% Depending on your tastes, you may want to make fonts of itemize environments slightly smaller
% \AtBeginEnvironment{itemize}{\small}

%% Provide the file name containing the sidebar contents as an optional parameter to \cvsection.
%% You can always just use \marginpar{...} if you do
%% not need to align the top of the contents to any
%% \cvsection title in the "main" bar.
\cvsection[sample-p1sidebar]{Professional Experience}

\cvevent{Telecommunication Engineering Intern}{Qualcomm}{Apr 2019 - Dec 2019}{São Paulo, Brasil}
\begin{itemize}
\item 4G and 5G Protocol Analysis and device tests.
\end{itemize}

\divider

\cvevent{Engineering Intern}{Intel Corporation}{Feb 2018 - Feb 2019}{Munich, Germany}
\begin{itemize}
\item Integration, verification and hardware solutions for G.Fast, VDSL and ADSL modem chips (Home Connected Division).
\item Support for engineers in test definition, test development and test execution.
\item Use of measurement, testing and modification equipment such as spectrum analyzer, oscilloscopes, loop simulators and soldering stations.
\item Develop solutions to hardware problems.
\end{itemize}

\cvsection{Academics Experience}

\cvevent{Scholarship Holder}{Fundunesp/UNESP}{Jan 2016 - Dec 2018}{}
Study, development and implementation of codings 8b/10b and 64b/66b. Matlab coding was modeled (SIMULINK) and then implemented in an FPGA (Xilinx Kintex 7) using VHDL. 

\divider

\cvevent{Undergraduate Researcher}{São Paulo Research and Analysis Center (SPRACE)/UNESP}{Oct 2015 - Dec 2018}{}
Student / Researcher in the field of electronic instrumentation for high energy physics.

\divider

\cvevent{Voluntary}{PET Elétrica/UNESP}{Dec 2014 - Mar 2017}{}
The main activity developed in the group was the Project Workshop in which I was leader of the activity. In this activity electronic projects were developed along with the freshmen of the course.

\medskip

%\cvsection{A Day of My Life}

% Adapted from @Jake's answer from http://tex.stackexchange.com/a/82729/226
% \wheelchart{outer radius}{inner radius}{
% comma-separated list of value/text width/color/detail}
%\wheelchart{1.5cm}{0.5cm}{%
%  6/8em/accent!30/{Sleep,\\beautiful sleep},
%  3/8em/accent!40/Hopeful novelist by night,
%  8/8em/accent!60/Daytime job,
%  2/10em/accent/Sports and relaxation,
%  5/6em/accent!20/Spending time with family
%}

%\clearpage
%\cvsection[page2sidebar]{Publicações}
\cvsection{Publications}

\nocite{*}

\printbibliography[heading=pubtype,title={\printinfo{\faBook}{Books}},type=book]

%\divider

\printbibliography[heading=pubtype,title={\printinfo{\faFileTextO}{Artigos}}, type=article]
\printinfo{\faFileTextO}{}Modeling and Implementation in FPGA of 8b/10b Encoding. SIIM/SPS. Nov 2017. Available in: <www.eventos.ufabc.edu.br/siimsps/files/id14.pdf>

\divider

\printinfo{\faFileTextO}{}Robustness Analysis and State Machine Modeling of 8b/10b Encoding. ERMAC. May 2017. Available in: <www.fc.unesp.br/Home/Departamentos/Matematica/ermac/caderno\-ermac\_2017.pdf>
%\divider

\printbibliography[heading=pubtype,title={\printinfo{\faGroup}{Conference Proceedings}},type=inproceedings]

%% If the NEXT page doesn't start with a \cvsection but you'd
%% still like to add a sidebar, then use this command on THIS
%% page to add it. The optional argument lets you pull up the
%% sidebar a bit so that it looks aligned with the top of the
%% main column.
% \addnextpagesidebar[-1ex]{page3sidebar}


\end{document}
